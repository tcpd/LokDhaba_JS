\section{User Feedback}
To get a sense of the usability of LokDhaba, we conducted a survey with expert political scientists and journalists who have used LokDhaba\footnote{Susmeet Jain and Keshav Joshi led the pilot study.}, and received 13 responses. Most users report using LokDhaba for research purposes (63.6\%). Unsurprisingly, close to half of our respondents use the platform around the time of an election (45.5\%). They report downloading more charts (90\%) than maps (54.5\%). Most users prefer working on LokDhaba on a computer rather than a mobile device (81.8\% and 36.4\% respectively). Most users report that it is easy for them to browse the data (81.9\%). However, only 63.6\% of the users agreed that the visualizations are effective and only 54.5\% report that the incumbency profile is intuitive.

Google Analytics statistics for the period May 2019-2020 (when there was a national election) showed that there were 10,284 unique users, with 15,383 sessions. Most of the traffic comes to LokDhaba directly (73.95\%), followed by organic search (12.42\%), followed by social channels like Twitter or Facebook (10.66\%), and referrals from other media sources (3\%). 89\% of users are from India, 5\% from the United States, and 2\% from the United Kingdom. The rest are from Singapore, Canada, France, U.A.E, Germany and a few other parts of the world.

\section{Conclusions and Impact}

In this paper, we have described the challenges encountered in assembling and maintaining a high-value longitudinal dataset. We have proposed some solutions, including new tools and best practices to meet these challenges that we hope will be useful in other domains as well.

Apart from research contributions, LokDhaba has had substantial impact on the state of the practice. We describe some of these impact areas below.

\subsection{Application}

The first contribution to the practice is that of an open source application with a simple user interface to browse and visualize electoral data. The data as well as the web application is freely available for download. The LokDhaba application code is maintained as a freely accessible open source git repository\cite{lokdhabaRepo} and can easily be used for other electoral datasets with a First Past the Post system.

\subsection{Study of Indian Elections}

LokDhaba makes several key contributions to the study of Indian elections. The first is that it pulls together information from multiple sources with varying formats. Building from ECI's statistical reports, LokDhaba has variables at both the constituency and candidate level. We have also integrated these with data on individuals' assets and crime records. In the future, we expect to bring together several indicators of social and political behavior, with the aim of enabling more research on political candidates.

Secondly, through the assignment of unique identifiers, we are able to track both politicians and political parties over time and space. Researchers can now ask a wider array of questions about the career trajectory of politicians, when they defect, to which party and their subsequent performance. 

Finally, data on bye-elections has been an important but neglected part of the electoral process. Bye-elections can now be empirically studied and evaluated along with the main elections.

\subsection{Training and Outreach}

LokDhaba also enables several initiatives which we broadly classify as outreach. The first is as a source of data for other data dissemination initiatives. We are aware of at least two repositories - Jaano India\footnote{https://jaanoindia.swaniti.org/} and the Constituency-Level Election Archive (CLEA) at the University of Michigan\footnote{http://www.electiondataarchive.org./} - that use the data provided by LokDhaba.

An unexpectedly pleasant use of LokDhaba has been for instruction in the classroom and for training inter-disciplinary students. LokDhaba is used to train and encourage students to think critically about data, spot anomalies, and analyze long-term patterns and trends. It is used during lectures in classes and in an annual summer school attended by undergraduate and post-graduate students, professionals and journalists. Our outreach demonstrates to our partners that building datasets is a non-trivial process and requires careful work in all phases of the data lifecycle.

\section{Future Work}

The LokDhaba dataset is being continually updated and is expanding in terms of integration with other datasets. We plan to build more map-based visualizations, including some that allow longitudinal comparison of results (e.g., vote share of parties across successive elections). We also need to improve our mapping infrastructure and get more authoritative sources of geo-spatial data. We welcome collaboration with groups that need to work with election results data.
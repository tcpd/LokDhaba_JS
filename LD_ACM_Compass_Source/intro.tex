
\section{Introduction}

The Indian elections are the largest in the world, currently involving an electorate of over 900 million people~\cite{quraishi2014undocumented}.  As specified in the Indian constitution, elections are held every five years at national (parliamentary), state and local levels, with a ``First Past The Post'' system. At the state and parliamentary level, the elections are conducted by the Election Commission of India (ECI) and the state election commissions in 29 states and 7 union territories. These elections generate a fascinating trove of data for understanding political trends and processes. However, it is difficult to obtain structured data for the outcomes of Indian elections in consistent formats over a long period of time. This inhibits fundamental questions political scientists would like to ask, such as: What is the pattern in voter turnout over time across the country? Are women participating more, or less, in the political process? Is the supply of political candidates increasing or decreasing over time? How many times does an aspirant for political office contest an election? Are these patterns correlated with caste, geography, age, gender or income?

In this paper, we discuss technical challenges we encountered in the acquisition and dissemination of Indian elections data that enables the answer to such questions. Over the past three years, we have set up a unique data processing pipeline to process the results of all Indian elections since 1962 and make them freely available to researchers, journalists and the general public. Our dataset includes cleaned and harmonized electoral results published by the ECI, new primary variables (such as unique identifiers for all individuals contesting Indian elections), and integration with datasets from other reliable sources such as the Association of Democratic Reforms (ADR), PRS Legislative Research and the Lok Sabha (the lower house of the Indian Parliament). Over this period, we have managed to keep our dataset up to date, and absorbed the results of over 20 state elections and one national election.

Our data processing pipeline and data dissemination portal is called LokDhaba\footnote{In Hindi, {\it Lok} means people, and {\it Dhaba} means a roadside eatery, often frequented by all sections of society}. The portal was designed by an inter-disciplinary team of computer scientists and political scientists, and is being used extensively by researchers\footnote{See {\href{https://tcpd.ashoka.edu.in/research-publications/}{https://tcpd.ashoka.edu.in/research-publications/}} for a list of research publications.} and journalists\footnote{See {\href{https://tcpd.ashoka.edu.in/press-articles/}{https://tcpd.ashoka.edu.in/press-articles/}} and {\href{https://tcpd.ashoka.edu.in/data-quoted/}{https://tcpd.ashoka.edu.in/data-quoted/}} for a list of articles published by various media houses.} working with Indian elections data. It was heavily cited by major news outlets across the world for the Indian parliamentary elections in 2019, including by the BBC and The New York Times. LokDhaba provides the cleaned and integrated data in a single portal at {\href{https://lokdhaba.ashoka.edu.in/}{https://lokdhaba.ashoka.edu.in/}}. The data can be freely downloaded and used for any purpose. Users can also visualize, bookmark and download data to analyze trends over time and space. 

Our research contributions in this paper are:
\begin{enumerate}
    \item We describe the data processing pipeline that we set up to acquire, process, check and disseminate data. Along the way, we designed solutions to technical challenges that are likely to be useful for any longitudinal dataset.
    \item We shed some light on how to deal with poorly organized government records in formats that drift over time. We illustrate how our resulting dataset was able to point out errors in the original dataset that might otherwise have remained undiscovered. 
    \item We describe ways to enhance the value of the dataset, for example, by resolving ambiguous terms, and by connecting it to other datasets. 
    \item Finally, we present an open source platform for political data dissemination and visualization that may be useful for any country with a First Past the Post system of electoral democracy. 
\end{enumerate}

This paper is structured as follows: We begin with a discussion of related work in the next section. In Section 3, we describe the data extraction process from the original data. Next, we describe how we organize and extend the dataset. In Section 5, we discuss the web application and visualization framework used to disseminate the data. Finally, we report some user feedback, and close by discussing the impact of our work.

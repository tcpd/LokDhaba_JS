

\section{Related Work}

Various efforts have been made in the past to work on Indian election results. Verniers and Jensenius provide a good primer on Indian election processes and the data generated by them, as well as the utility of this data in political science \cite{jensenius2017studying}. The primary source of data in this area is the set of verified statistical reports released by the ECI after every election \cite{StatReportECI}. Commercial services like Nielsen, \cite{nielsen1993nielsen}, I-PAC\cite{singhelection}, Datanet India \cite{electionsInIndia} and The Center for Monitoring Indian Economy \cite{center2000cmie} use this data to generate political insights for their customers. Lokniti \cite{bose1987state,shastri2009electoral}, a research unit of Center for Study of Developing Societies (CSDS) also digitizes election results along with state and national level surveys; however, their data is not openly available for downloading and analysis. While a few companies like India Votes \cite{rana2006india,indiavotes} publicly release party-wise or constituency-wise aggregates, their methods for obtaining, treating and updating data are opaque, and their data tends to be incomplete.

The Association for Democratic Reforms (ADR) digitizes mandatory affidavits filed by candidates and releases information on education, occupation, assets, criminal cases, and other individual level attributes on the MyNeta portal. This information on election candidates has led to a growth in the literature on corruption and elections in India~\cite{vaishnav2017crime,fisman2016financial,dash2004civil, bhavnani2012using}. However, ADR's data is not linked with the electoral performance of the corresponding individual and spans a shorter time range since the affidavits are available only since 2004. LokDhaba integrates the election results dataset from the ECI with ADR's affidavit database.

The Constituency-Level Elections Archive (CLEA)\cite{kollman2011constituency}, is the most comprehensive repository of detailed results for national elections from around the world. There is also the Socioeconomic High-resolution Rural-Urban Geographic Dataset on India (SHRUG) \cite{DVN/DPESAK_2019} dataset which contains indicators on demographic, socioeconomic, firm and political outcomes from 1990-2013. And lastly, individual scholars too have also released partial datasets on Indian elections \cite{bhavnani2015socio, brass1978indian, nooruddin2008unstable, blair1973minority, heath2018electoral, chandra2016democratic, kagan2015using}. 

While many of the above datasets have been used to answer questions of interest to specific researchers, the LokDhaba dataset is more extensive and has undergone rigorous data treatment and consistency checking. It encompasses a comprehensive range of variables; for example, it incorporates data about bye-elections, which are generally ignored by other datasets, and attempts to identify unique entities (such as individuals and parties) in the Indian political sphere. LokDhaba data is also freely available for any purpose. LokDhaba allows users to build their own data visualizations, or use it's API capabilities to use the data for any purpose.

% \begin{table}[h]
%   \caption{Comparison of features across open election data providers}
%   \label{tab:freq}
%   \begin{tabular}{l|ccc}
%     \toprule
%     Features & LokDhaba & CLEA & IndiaVotes \\
%     \midrule
%     Parl. elections& '62-'19 & '62-'19 & '62-'14 \\
%     VS time span& '62-'20 & NA & '77-'18 \\
%     Bye-Election data & $\surd$ &  & \\
%     Constituency maps& $\surd$ & $\surd$ & \\
%     Party identifiers& $\surd$ &  & \\
%     Candidate identifiers& $\surd$ &  & \\
%     Browse results& $\surd$ & & $\surd$ \\
%     Data visualization& $\surd$ & &  \\
%     Open source& $\surd$ &  & \\
%     Data download& $\surd$ & $\surd$ & $\surd$\\
%     API& $\surd$ &  & \\
%   \bottomrule
% \end{tabular}
% \end{table}





